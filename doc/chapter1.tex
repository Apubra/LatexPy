%%%% This template is customised for edited volumes published with Cambridge
%%%% University Press. It should be used in conjunction with the other files
%%%% included in cup_volumes.zip available from
%%%% http://www.michaelcuffaro.com/latex.shtml
%%%%
%%%% Note: Lorem ipsum text generated via http://www.lipsum.com/
%%%%
%%%% It is freely available for you to use.
%%%%
%%%% Copyright 2017, Michael Cuffaro.


%\renewcommand{\chapterauthor}{Nan Tucket}
%\renewcommand{\shortchaptertitle}{The greatest chapter on Earth}
\chapter{Reversible Logic Synthesis}
\label{ch:part1_chapter1}

\settocdepth{section}
\cftaddtitleline{toc}{section}{\emph{\chapterauthor}}{}
\settocdepth{chapter}

Reversible logic plays a vital  role  at present time and it has different areas for its applications, namely Low Power CMOS, Quantum Computing, Nanotechnology, Cryptography,
Optical Computing, DNA  Computing, Digital Signal  Processing (DSP), Quantum-dot Cellular Automata (QCA),  Digital Communications, and Computer Graphics. It is not possible to realize quantum computing without implementation of reversible logic. The main purposes of designing reversible logic circuits are to decrease quantum cost, depth of the circuits, and the number of garbage outputs.%
This chapter explains the basic reversible logic gates for more complex system which may have reversible circuits as a primitive component and can execute complicated operations using quantum computers. The reversible circuits form the basic building block of quantum computers as all quantum operations are reversible. This chapter presents the information related to the primitive reversible gates and helps researches in designing higher complex computing circuits using reversible gates.
 
 
\section{Reversible Logic}
 In this section, basic definitions and ideas related to reversible logic are presented. Formal definitions of reversible gate, garbage output and the popular reversible gates along with their input-output vectors are presented here.
 Illustrative figures and examples are also included in respective discussions.
 
\section{Reversible Function}
The multiple output Boolean Function $F(x_1, x_2, ...,x_n)$ of \textit{n} Boolean variables is called reversible if:
\begin{enumerate}
	\item The number of outputs is equal to the number of inputs
	\item Any output pattern has a unique pre-image
\end{enumerate}


In other words, the functions which perform permutations of the set of input vectors are referred to as reversible functions.

\section{Reversible Logic Gate}
Reversible logic has unique mapping between input and output bit pattern. A unit logic entity is represented as gate. The gates or circuits that do not loose information are called reversible gates or circuits.

\begin{property}\textnormal{
A reversible circuit is such a circuit in which the number of input and the number of output is equal and there is one-to-one mapping between input and output vectors.}
\end{property}
 
Let us consider the gate shown in Figure~\ref{fig:p1_c1_fig1}. According to the definition, the gate is a reversible gate, because it has \textit{k} number of inputs and k number of outputs and the gate is known as $k \times k$ Reversible Gate. Without the NOT gate, classical logic gates are called irreversible, since they cannot determine the input vector states from the output vector states uniquely.
\begin{figure}[H]
	\centering
	\includegraphics[width=8cm, height=5cm]{chapters/part1/chapter1/fig1.eps}
	\caption{A $k \times k$ Reversible Gate}
	\label{fig:p1_c1_fig1}
\end{figure}

\begin{example}\textnormal{
	There can be any number of dimensions for a reversible gate, but lower dimension is always preferable for designing efficient circuits. The popular reversible gates, Feynman Gate (FG), Toffoli Gate (TG), Peres Gate (PG), Fredkin Gate (FRG), Feynman Double Gate (F2G) and New Fault Tolerant (NFT), are shown in Figure~\ref{fig:p1_c1_fig2}}
\end{example}
\begin{figure}[H]
	\centering
	\begin{subfigure}[b]{0.30\textwidth}
		\centering
		\includegraphics[width=\textwidth]{chapters/part1/chapter1/fig2_a.eps}
		\caption{Feynman Gate}
		\label{fig:p1_c1_fig2_a}
	\end{subfigure}
	\begin{subfigure}[b]{0.30\textwidth}
		\centering
		\includegraphics[width=\textwidth]{chapters/part1/chapter1/fig2_b.eps}
		\caption{Peres Gate}
		\label{fig:p2_c1_fig2_b}
	\end{subfigure}
	\begin{subfigure}[b]{0.30\textwidth}
		\centering
		\includegraphics[width=\textwidth]{chapters/part1/chapter1/fig2_c.eps}
		\caption{Toffoli Gate}
		\label{fig:p1_c1_fig2_c}
	\end{subfigure}

\begin{subfigure}[b]{0.30\textwidth}
	\centering
	\includegraphics[width=\textwidth]{chapters/part1/chapter1/fig2_d.eps}
	\caption{Feynman Double Gate}
	\label{fig:p1_c1_fig2_d}
\end{subfigure}
\begin{subfigure}[b]{0.30\textwidth}
	\centering
	\includegraphics[width=\textwidth]{chapters/part1/chapter1/fig2_e.eps}
	\caption{New Fault Tolerant Gate}
	\label{fig:p1_c1_fig2_e}
\end{subfigure}
\begin{subfigure}[b]{0.30\textwidth}
	\centering
	\includegraphics[width=\textwidth]{chapters/part1/chapter1/fig2_f.eps}
	\caption{Fredkin Gate}
	\label{fig:p1_c1_fig2_f}
\end{subfigure}
	\caption{Popular Reversible Gates}
	\label{fig:p1_c1_fig2}
\end{figure}

\section{Garbage Outputs}

The  output (outputs)  of  a reversible  gate which is  (are)  not used as input to other gate or  the  output  (outputs) which is (are) not treated as a primary output is  (are) called garbage output (outputs). The unutilized outputs 
from a gate are called garbage outputs. Heavy price is paid
off for every garbage output. So, the less number of garbage output  is desired  for any circuit design.


%Every gate's output that is not used as input to other gate or as a primary output is called garbage output. The unutilized outputs from a gate are called garbage. Heavy price is paid off for every garbage output. So, it should always keep in mind that, the less number of garbage is quite good for any circuit design.

\begin{example}\textnormal{
	When a Feynman gate (FG) is used for Ex-OR (Exclusive-OR) operation of two inputs, extra one output is generated at the output part of the FG in addition to the Ex-OR output. This additional output is known as garbage output. In Figure~\ref{fig:p1_c1_fig3}, the garbage output of a gate is shown. Here, A is the garbage output.}
\end{example}

\begin{figure}[H]
	\centering
	\includegraphics[width=0.4\textwidth]{chapters/part1/chapter1/fig3.eps}
	\caption{A $k \times k$ Reversible Gate}
	\label{fig:p1_c1_fig3}
\end{figure}

\section{Constant Inputs}
Constant inputs are the inputs of a reversible gate (or circuit) that are either set to 0 or 1.

\begin{example}\textnormal{
	If the complement of the input A from Figure~\ref{fig:p1_c1_fig3} is needed, then \textit{B} is set to 1 and $Q = A$.}
\end{example}

\section{Quantum Cost}
%Quantum theory is a gigantic research field deeply related to crucial applications as DNA technologies, nanotechnologies and optical computing etc. Reversible quantum logic plays an important role in quantum computing.

%\begin{property}\textnormal{
	The quantum cost of a circuit is the total number of $2 \times 2$ quantum primitives are used to realize corresponding quantum circuit. Basically, the quantum primitives are matrix operations, which are applied on qubits state.%}
%\end{property} 



\begin{example}\textnormal{
The quantum realization of reversible Fredkin (FRG) gate is shown in  Figure~\ref{fig:p1_c1_fig5_1}. Each quantum Ex-OR gate and quantum V or V$^+$ gate requires 1 (one) quantum cost. The reversible FRG gate has four quantum Ex-OR gates, two quantum V gates and one quantum V$^+$ gate. So, the quantum cost of reversible FRG gate seems 7 (seven). But, we know if a quantum Ex-OR gate and a quantum V or  V$^+$ gate exists angularly (denoted by angular  box), then the  quantum cost treats as 1. From the figure, we see that, there exists two angular boxes and each angular box treats as 1 quantum cost. As a result, the total quantum cost of reversible FRG gate is 5 (five).}
\end{example}

\begin{figure}[H]
	\centering
	\includegraphics[width=0.5\textwidth]{chapters/part1/chapter1/fig5_1.eps}
	\caption{Quantum Realization of Reversible Fredkin (FRG) Gate}
	\label{fig:p1_c1_fig5_1}
\end{figure}
\begin{example}\textnormal{
	The cost of all $2 \times 2$ gates are same and it is 1. For $1 \times 1$ gate, the cost is 0. Every circuit can be constructed from those $1 \times 1$ and $2 \times 2$ quantum primitives and the cost of circuit is the total sum of required $2 \times 2$ gates.}
\end{example}
\section{Delay}
The delay of a logic circuit is the maximum number of gates in a path from any input line to any output-line. The definition is based on two assumptions: (i) Each gate performs computation in one unit time and (ii) all inputs to the circuit are available before the computation begins.


\begin{example}\textnormal{
	The delay of each $1 \times 1$ and $2 \times 2$ reversible gate is taken as unit delay 1. Any $3 \times 3$ reversible gate can be designed from $1 \times 1$ reversible gates and $2\times 2$ reversible gates, such as CNOT gate, Controlled-V and Controlled-V$^+$ gates (V is a square-root-of NOT gate and V$^+$ is its hermitian). Thus, the delay of a $3 \times 3$ reversible gate can be computed by calculating its logical depth when it is designed from smaller $1 \times 1$ and $2 \times 2$ reversible gates.}
\end{example}

\section{Power}
Power of a gate is defined by the energy. Energy of a basic quantum gate is 142.3 meV. Quantum circuits can be implemented with the basic quantum gates and the number of quantum gates depends on the number of basic quantum gates needed to realize it. That means  the total number of required quantum gates in the quantum representation of a reversible quantum circuit or gate. So, the power of a reversible gate can be defined as follows: 

\hspace{1cm}{\textit{Power = Number of Quantum Gates $\times$ Energy of a Basic Quantum Gate}}

\begin{example}\textnormal{
	Figure~\ref{fig:p1_c1_fig7_1} shows the quantum realization of the reversible HNG gate. From this figures, it is seen that the quantum realization of reversible HNG gate requires total six quantum gates. So, the power of the reversible HNG gate is (6 $\times$ 142.3) meV = 853.8 meV, where the number of quantum gates of HNG circuit is 6.}
\end{example}


\begin{figure}[H]
	\centering
	\includegraphics[width=0.8\textwidth]{chapters/part1/chapter1/fig7_1.eps}
	\caption{The Quantum Representation of Reversible HNG Gate}
	\label{fig:p1_c1_fig7_1}
\end{figure}

\section{Area}
Area of a  reversible gate is defined by the feature size. This size varies according to the number of quantum gates. The size of the basic quantum gates ranges 50 A$^0$-300 A$^0$. The Angstrom (A$^0$) is a unit equal to 10-10 m (one ten-billionth of a meter) or 0.1 nm. So, the area of a reversible gate can be defined as follows: 

\hspace{1.51cm}{\textit{Area = Number of Quantum Gates $\times$  Size of  a Basic Quantum Gate}}

\begin{example}\textnormal{
	Figure~\ref{fig:p1_c1_fig7_1} shows the quantum realization of the reversible HNG gate. From this figures, it is seen that the quantum realization of reversible HNG gate requires total six quantum gates. So, the area of the reversible HNG gate is ((50 $\times$ 6) A$^0$ - (300 $\times$ 6)A$^0$) = (300 A$^0$ - 1800 A$^0$), where the number of quantum gates of HNG circuit is 6.}
\end{example} 


%\section{Flexibility}
%This refers to the universality of a reversible logic gate in realizing more functions.

\section{Hardware Complexity}
The Hardware Complexity of a reversible logic circuit specifies the total number of Ex-OR operations, NOT operations, and AND operations used in the circuit. Consequently, the hardware complexity can be determined using the following equation:

\counterwithin{equation}{section}
\begin{equation}
T=\alpha+\beta+\delta
\end{equation}

\noindent where\\
$T$ = Hardware Complexity  (Total  Logical  Operations)\\ 
$\alpha$ = A two input EX-OR gate logical operation \\
$\beta$ = A two input AND gate logical operation \\
$\delta$ = A  NOT gate logical  operation  

\begin{example}\textnormal{
	Figure~\ref{fig:p1_c1_fig9_1} shows the block diagram of  a reversible Fredkin (FRG) gate. The figure describes that there is only one NOT operation, two EX-OR operations and four AND operations. So, the Hardware Complexity of the reversible FRG gate is $T= 2\alpha+4\beta+1\delta$}
\end{example}
\begin{figure}[H]
	\centering
	\includegraphics[width=0.4\textwidth]{chapters/part1/chapter1/fig9_1.eps}
	\caption{Block Diagram of the Reversible FRG Gate}
	\label{fig:p1_c1_fig9_1}
\end{figure}

\section{Quantum Gate Calculation Complexity}
The Quantum Gate Calculation Complexity of the quantum  representation  of a reversible circuit specifies the total number of quantum gates (NOT gates, CNOT gates, and Controlled-V (Controlled-V$^+$) gates) used in the quantum representation of a reversible circuit. Consequently, the Quantum Gate Calculation Complexity can be determined using the following equation:


\begin{equation}
Q=\rho+\sigma+\Omega
\end{equation}
\noindent where\\
$Q$ = Quantum Gate Calculation Complexity\\
$\rho$ = A quantum NOT gate \\
$\sigma$ = A quantum CNOT gate   \\
$\Omega$ = A quantum Controlled-V (Controlled-V$^+$) gate


\begin{example}\textnormal{
Figure~\ref{fig:p1_c1_fig10_1} shows the quantum representation of  a  reversible Fredkin (FRG) gate. The figure describes that there is only one NOT operation, four  quantum CNOT operations and three  quantum Controlled-V (Controlled-V$^+$) operations. So, the  Quantum Gate Calculation Complexity of the reversible FRG gate is $Q =1\rho +4\sigma + 3\Omega$.}
\end{example}


\begin{figure}[H]
	\centering
	\includegraphics[width=0.5\textwidth]{chapters/part1/chapter1/fig10_1.eps}
	\caption{Quantum  Representation  of  a  Reversible FRG Gate}
	\label{fig:p1_c1_fig10_1}
\end{figure}



\section{Fan-Out}
Fan-out is a term that defines the maximum number of inputs in which the output of a single logic gate can be fed. The fan-out of any reversible circuit is 1.
\begin{example}\textnormal{
	The fan-out of any reversible circuit is 1.}
\end{example}


\section{Self-Reversible}
A gate is said to be self-reversible, if its dual combination is the same as itself.

\begin{example}\textnormal{
	n Figure~\ref{fig:p1_c1_fig4}, there are two Toffoli gates which are in the cascading form. If the outputs of the first Toffoli gate are fed to the input of the second Toffoli gate, then the output of the second Toffoli gate is equal to the input of the first Toffoli gate.
Here the outputs of first gate are \textit{P, Q} and \textit{R}, where \textit{P = A}, \textit{Q = B}, and \textit{R = AB} $\oplus$ \textit{C}. Then the outputs of second gate are \textit{X, V} and \textit{Z}, where \textit{X = A, Y = B} and \textit{Z = AB $\oplus$ AB$\oplus$ C = 0 $\oplus$ C = C.}}
\end{example}



\begin{figure}[h]
	\centering
	\includegraphics[width=0.6\textwidth]{chapters/part1/chapter1/fig4.eps}
	\caption{Toffoli Gates as Self-Reversible}
	\label{fig:p1_c1_fig4}
\end{figure}

\section{Reversible Computation}
In a reversible circuit, correct output is found by applying correct input instance and controlling one or more inputs if needed. Feynman gate (FG) is already presented to illustrate the idea of garbage output, Feynman gate is 2 $\times$ 2 reversible gate where inputs are A, B and corresponding functions are \textit{P = A, Q = A $\oplus$ B}. Feynman gate is used here to show how to control input to produce expected output. Both the inputs A and B are used as control inputs and their impact on output is shown below.

\textit{A} as control input:
	\begin{itemize}
		\item { } For \textit{A = 0}. Output \textit{P = 0 }and \textit{Q = B},
		\item { } For \textit{A = 1}, output \textit{P = 1} and \textit{Q = B'}.
		
	\end{itemize}

\textit{B} as control input:
\begin{itemize}
	\item For \textit{B = 0}, output \textit{P = A} and \textit{Q = A}.
	\item For \textit{B = 1}, output \textit{P = A} and\textit{ Q = A'}.
	
\end{itemize}

It is instructive to note that when \textit{B} is used as control input and \textit{B = 0}, both the outputs \textit{P = B} and \textit{Q = A}. This is, by controlling \textit{B}; copies of \textit{A} can easily create. This circuit can be easily used as a copying circuit.

\section{Area}
The area of a logic circuit is the summation of individual areas of each gate of the circuit. Suppose a reversible circuit is consisted of \textit{n} reversible gates. Area of those \textit{n} gates are $a_1, a_2,\cdots, a_n$. Then by using above definition area, denoted by \textit{A}, of that circuit is,
\begin{equation*}
A=\sum_{i=1}^{n}a_i
\end{equation*}

Given the above definition the area of a circuit can be calculated easily by obtaining area of each individual gate using CMOS 45 nm Open Cell Library and Synopsis Design Compiler. 

Area of a gate can also be defined by the feature size. This size varies according to the number of quantum gates. As the basic quantum gates are fabricated with quantum dots with the size ranges from several to tens of nanometers (10-9 m) in diameter, the size of the basic quantum gates ranges $50A^o-300A^o$. The Angstrom $(A^o)$ is a unit equal to 10-10m (one ten-billionth of a meter) or 0.1 nm. Its symbol is the Swedish letter $A^o$. Quantum circuits can be implemented with the basic quantum gates and the number of quantum gates depends on the number of basic quantum gates needed to implement it. So, the area of a gate can be defined as follows: Area = Number of Quantum Gates $\times$ Size of Basic Quantum Gates. 

\section{Design Constraints for Reversible Logic Circuits}
The following are the important design constraints for reversible logic circuits.

\begin{itemize}
	\item Reversible logic gates do not allow fan-outs.
	\item The reversible logic circuits should have minimum number of reversible gates.
	\item Reversible logic circuits should have minimum quantum cost.
	\item The design can be optimized so as to produce minimum number of garbage outputs.
	\item The reversible logic circuits must use minimum number of constant inputs.
	\item The reversible logic circuits must use a minimum logic depth or gate levels.
	\item Reversible logic circuits should have minimum area and power.
	\item The reversible logic circuits must use minimum hardware complexity and minimum quantum gate calculation complexity.
\end{itemize}

\section{Quantum Analysis of Different Reversible Logic Gates}
Calculating Quantum Cost of reversible circuit is always an interesting one. Quantum circuits, DNA technologies, nano-technologies and optical computing are the most common applications of quantum theory. Every reversible gate can be calculated in terms of quantum cost and hence the reversible circuits can be measured in terms of quantum cost. Reducing the quantum cost from reversible circuit is always a challenging one and works are still going on in this area. In this section, the quantum equivalent diagram of some popular reversible gate is presented.

\begin{property}\textnormal{
	The quantum cost of every 2$\times$2 gate is the same. It can be easily assumed that 1$\times$1 gate cost nothing, since it can be always included to arbitrary 2$\times$2 gate that precedes or follows it. Thus, in first approximation, every permutation quantum gate will be built from 1$\times$1 and 2$\times$2 quantum primitives and its cost calculated as a total sum of 2$\times$2 gates used. All gates of the form 2$\times$2 has equal quantum cost, and the cost is unity.}
\end{property}

\subsection{Reversible NOT Gate(Feynman Gate)}

\begin{example}\textnormal{
	A 2$\times$2 Feynman gate is also called CNOT. This gate is one through because it passes one of its inputs. Every linear reversible function can be built by using only 2$\times$2 Feynman gates and inverters. Since this is a 2$\times$2 gate, the quantum cost is 1. Quantum equivalent circuit of Feynman gate is shown in Figure~\ref{fig:p1_c1_fig5}.}
\end{example}

\begin{figure}[h]
	\centering
	\includegraphics[width=0.2\textwidth]{chapters/part1/chapter1/fig5.eps}
	\caption{Quantum Cost Calculation of Feynman Gate}
	\label{fig:p1_c1_fig5}
\end{figure}

\subsection{Toffoli Gate}
	Figure~\ref{fig:p1_c1_fig6} shows the equivalent quantum realization of three input Toffoli gate. The cost of, the Toffoli gate is five 2 $\times$ 2 gates, or simply 5. In Figure~\ref{fig:p1_c1_fig6}, V is a square-root of NOT gate and V$^+$ is its hermitian. Thus VV$^+$ creates a unitary matrix of NOT gate and VV$^+$ = I (an identity matrix, describing just a quantum wire).
	
	\begin{figure}[h]
		\centering
		\includegraphics[width=0.6\textwidth]{chapters/part1/chapter1/fig6.eps}
		\caption{Quantum Circuit of Toffoli Gate}
		\label{fig:p1_c1_fig6}
	\end{figure}

\subsection{Fredkin Gate}
The cost of Fredkin gate is the same as the cost of Toffoli gate. Toffoli gate includes a single Davio gate, while the Fredkin gate includes two multiplexers. Quantum equivalent Toffoli gate is shown in Figure~\ref{fig:p1_c1_fig6}. Each dotted rectangles in Figure~\ref{fig:p1_c1_fig7} is equivalent to a 2 $\times$ 2 Feynman gate and so the cost is 1 for the particular case.

	\begin{figure}[h]
	\centering
	\includegraphics[width=0.6\textwidth]{chapters/part1/chapter1/fig7.eps}
	\caption{Quantum Circuit of Fredkin Gate}
	\label{fig:p1_c1_fig7}
\end{figure}

\subsection{Peres Gate}
	This gate can be realized with cost 4. It is just like a Toffoli gate but without the last Feynman gate from right. This is the cheapest realization of a complete (universal) 3 $\times$ 3 permutation gate. Figure~\ref{fig:p1_c1_fig8} shows the quantum realization of Peres gate.
	\\
	\begin{figure}[h]
		\centering
		\includegraphics[width=0.6\textwidth]{chapters/part1/chapter1/fig8.eps}
		\caption{Quantum Circuit of Peres Gate}
		\label{fig:p1_c1_fig8}
	\end{figure}

\section{Summary}
Maxwell's demon and Szilard's analysis of the demon first suggested the connection between a single degree of freedom (one bit) and a minimum quantity of entropy. In 1950s, this connection had been popularly interpreted to mean that computation must dissipate a corresponding minimum amount of energy during every elemental act of computation. Landauer later recognized that energy dissipation is only unavoidable when information is destroyed. Bennett and Toffoli first realized that a reversible computation, in which no information is destroyed, may dissipate arbitrarily small amounts of energy. The reversible circuits form the basic building block of quantum computers. This chapter presents some reversible gates. This chapter will help researches/designers in designing higher complex computing circuits using reversible gates. It can further be extended towards the digital design  development using reversible logic circuits which are helpful in quantum computing, low power CMOS, nanotechnology, cryptography, optical computing, DNA computing, digital signal processing (DSP), quantum dot cellular automata, communication, computer graphics.

